\documentclass[pdftex,12pt,a4paper]{report}
\usepackage[pdftex]{graphicx}
\usepackage[francais]{babel} 
\usepackage[utf8]{inputenc}
\usepackage[T1]{fontenc}
\usepackage[top=2cm, bottom=2cm, left=3cm, right=3cm]{geometry}
\usepackage{graphicx}
\usepackage[toc,page]{appendix} 
\usepackage{pdfpages}
\usepackage{color}
\usepackage{chngpage}
\usepackage{fancyhdr}
\usepackage{fancybox}
\usepackage{lastpage}

\setcounter{tocdepth}{3}
\setcounter{secnumdepth}{3} 
\setlength{\headheight}{14pt}
\renewcommand\headrulewidth{1pt}
\fancyhead[L]{Signature}
\fancyhead[R]{EPITA}

\renewcommand\footrulewidth{1pt}
\fancyfoot[L]{Signature}
\fancyfoot[R]{Page \thepage/\pageref{LastPage}}

\newcommand{\HRule}{\rule{\linewidth}{0.5mm}}

\pagestyle{fancy}

\begin{document}
\begin{titlepage}
\begin{center}
\textsc{\LARGE Signature}\\[1.5cm]

% Title
\HRule \\[0.4cm]
{\huge \bfseries Signature}\\[0.4cm]
\HRule \\[1.5cm]
\end{center}

% Author and supervisor
\begin{minipage}{0.4\textwidth}
	\begin{flushleft} \large
		\emph{Auteurs:}\\
			Victor \textsc{Degliame} \\
			Tanguy \textsc{Abel} \\
			Thibault \textsc{Lapassade} \\
			Florian \textsc{Thomassin} \\
	\end{flushleft}
\end{minipage}
\begin{minipage}{0.4\textwidth}
	\begin{flushright} \large
		\emph{Relecture:} \\
              Reda \textsc{Dehak}
	\end{flushright}
\end{minipage}
\end{titlepage}

\tableofcontents

\section{Dimension fractale}
La signature d'un individu est caractérisé par de multiple éléments qui ne sont pas discernable a l'œil nue.\\

Pour extraire ces caractéristiques, on peut calculer la dimension fractale qui d'après la définition permet d'obtenir une grandeur qui va traduire la façon dont un ensemble fractal de remplir l'espace.\\

De plus, selon Mandelbrot, la dimension fractal permet de quantifier la complexité d'une courbe.
C'est cette dernière qui nous intéressé pour la reconnaissance de signature.\\

Il existe de multiple définition de cette dimension (Hausdorff, Homothetie, Minkowski-Bouligand).C'est cette dernière que nous utilisons, elle est aussi appelle "Box-Counting".\\

Cette technique est la plus utilisé et permet de mesurer numériquement la dimension fractale. Cela se fait via la formule suivante:
\begin{center}
\Ovalbox{$D_{box} = \lim\limits_{\epsilon \rightarrow 0}(\frac{log N(\epsilon)}{log N(1/\epsilon)})$ avec $N(\epsilon)$ nombre d'élément de diamètre $\epsilon$}
\end{center}

La méthode de Minkowski-Bouligand va recouvrir et créer un réseau de cube N fois sur la forme que l'on va analyser, N allant jusqu'à $+\infty$. En étudiant le comportement de ce réseau on va pouvoir calculer la dimension fractale.

\end{document}